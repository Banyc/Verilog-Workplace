\PassOptionsToPackage{unicode=true}{hyperref} % options for packages loaded elsewhere
\PassOptionsToPackage{hyphens}{url}
%
\documentclass[UTF8]{ctexart}
\usepackage{lmodern}
\usepackage{amssymb,amsmath}
\usepackage{ifxetex,ifluatex}
\usepackage{fixltx2e} % provides \textsubscript
\ifnum 0\ifxetex 1\fi\ifluatex 1\fi=0 % if pdftex
  \usepackage[T1]{fontenc}
  \usepackage{textcomp} % provides euro and other symbols
\else % if luatex or xelatex
  \usepackage{unicode-math}
  \defaultfontfeatures{Ligatures=TeX,Scale=MatchLowercase}
\fi
% use upquote if available, for straight quotes in verbatim environments
\IfFileExists{upquote.sty}{\usepackage{upquote}}{}
% use microtype if available
\IfFileExists{microtype.sty}{%
\usepackage[]{microtype}
\UseMicrotypeSet[protrusion]{basicmath} % disable protrusion for tt fonts
}{}
\IfFileExists{parskip.sty}{%
\usepackage{parskip}
}{% else
\setlength{\parindent}{0pt}
\setlength{\parskip}{6pt plus 2pt minus 1pt}
}
\usepackage{hyperref}
\hypersetup{
            pdfborder={0 0 0},
            breaklinks=true}
\urlstyle{same}  % don't use monospace font for urls
\usepackage{color}
\usepackage{fancyvrb}
\newcommand{\VerbBar}{|}
\newcommand{\VERB}{\Verb[commandchars=\\\{\}]}
\DefineVerbatimEnvironment{Highlighting}{Verbatim}{commandchars=\\\{\}}
% Add ',fontsize=\small' for more characters per line
\newenvironment{Shaded}{}{}
\newcommand{\AlertTok}[1]{\textcolor[rgb]{1.00,0.00,0.00}{\textbf{#1}}}
\newcommand{\AnnotationTok}[1]{\textcolor[rgb]{0.38,0.63,0.69}{\textbf{\textit{#1}}}}
\newcommand{\AttributeTok}[1]{\textcolor[rgb]{0.49,0.56,0.16}{#1}}
\newcommand{\BaseNTok}[1]{\textcolor[rgb]{0.25,0.63,0.44}{#1}}
\newcommand{\BuiltInTok}[1]{#1}
\newcommand{\CharTok}[1]{\textcolor[rgb]{0.25,0.44,0.63}{#1}}
\newcommand{\CommentTok}[1]{\textcolor[rgb]{0.38,0.63,0.69}{\textit{#1}}}
\newcommand{\CommentVarTok}[1]{\textcolor[rgb]{0.38,0.63,0.69}{\textbf{\textit{#1}}}}
\newcommand{\ConstantTok}[1]{\textcolor[rgb]{0.53,0.00,0.00}{#1}}
\newcommand{\ControlFlowTok}[1]{\textcolor[rgb]{0.00,0.44,0.13}{\textbf{#1}}}
\newcommand{\DataTypeTok}[1]{\textcolor[rgb]{0.56,0.13,0.00}{#1}}
\newcommand{\DecValTok}[1]{\textcolor[rgb]{0.25,0.63,0.44}{#1}}
\newcommand{\DocumentationTok}[1]{\textcolor[rgb]{0.73,0.13,0.13}{\textit{#1}}}
\newcommand{\ErrorTok}[1]{\textcolor[rgb]{1.00,0.00,0.00}{\textbf{#1}}}
\newcommand{\ExtensionTok}[1]{#1}
\newcommand{\FloatTok}[1]{\textcolor[rgb]{0.25,0.63,0.44}{#1}}
\newcommand{\FunctionTok}[1]{\textcolor[rgb]{0.02,0.16,0.49}{#1}}
\newcommand{\ImportTok}[1]{#1}
\newcommand{\InformationTok}[1]{\textcolor[rgb]{0.38,0.63,0.69}{\textbf{\textit{#1}}}}
\newcommand{\KeywordTok}[1]{\textcolor[rgb]{0.00,0.44,0.13}{\textbf{#1}}}
\newcommand{\NormalTok}[1]{#1}
\newcommand{\OperatorTok}[1]{\textcolor[rgb]{0.40,0.40,0.40}{#1}}
\newcommand{\OtherTok}[1]{\textcolor[rgb]{0.00,0.44,0.13}{#1}}
\newcommand{\PreprocessorTok}[1]{\textcolor[rgb]{0.74,0.48,0.00}{#1}}
\newcommand{\RegionMarkerTok}[1]{#1}
\newcommand{\SpecialCharTok}[1]{\textcolor[rgb]{0.25,0.44,0.63}{#1}}
\newcommand{\SpecialStringTok}[1]{\textcolor[rgb]{0.73,0.40,0.53}{#1}}
\newcommand{\StringTok}[1]{\textcolor[rgb]{0.25,0.44,0.63}{#1}}
\newcommand{\VariableTok}[1]{\textcolor[rgb]{0.10,0.09,0.49}{#1}}
\newcommand{\VerbatimStringTok}[1]{\textcolor[rgb]{0.25,0.44,0.63}{#1}}
\newcommand{\WarningTok}[1]{\textcolor[rgb]{0.38,0.63,0.69}{\textbf{\textit{#1}}}}
\usepackage{graphicx,grffile}
\makeatletter
\def\maxwidth{\ifdim\Gin@nat@width>\linewidth\linewidth\else\Gin@nat@width\fi}
\def\maxheight{\ifdim\Gin@nat@height>\textheight\textheight\else\Gin@nat@height\fi}
\makeatother
% Scale images if necessary, so that they will not overflow the page
% margins by default, and it is still possible to overwrite the defaults
% using explicit options in \includegraphics[width, height, ...]{}
\setkeys{Gin}{width=\maxwidth,height=\maxheight,keepaspectratio}
\setlength{\emergencystretch}{3em}  % prevent overfull lines
\providecommand{\tightlist}{%
  \setlength{\itemsep}{0pt}\setlength{\parskip}{0pt}}
\setcounter{secnumdepth}{0}
% Redefines (sub)paragraphs to behave more like sections
\ifx\paragraph\undefined\else
\let\oldparagraph\paragraph
\renewcommand{\paragraph}[1]{\oldparagraph{#1}\mbox{}}
\fi
\ifx\subparagraph\undefined\else
\let\oldsubparagraph\subparagraph
\renewcommand{\subparagraph}[1]{\oldsubparagraph{#1}\mbox{}}
\fi

% set default figure placement to htbp
\makeatletter
\def\fps@figure{htbp}
\makeatother


\date{}

\begin{document}

\hypertarget{ux5b58ux50a8ux5668ux5b9eux73b0-ux5b58ux50a8ux5668ux8bfbux5199}{%
\section{存储器实现 \&
存储器读写}\label{ux5b58ux50a8ux5668ux5b9eux73b0-ux5b58ux50a8ux5668ux8bfbux5199}}

\hypertarget{ux5b9eux9a8cux76eeux7684}{%
\subsection{实验目的}\label{ux5b9eux9a8cux76eeux7684}}

目标:存储器读写。自选合适的方式选择地址写入32位数据,并能按任意地址读出存储器数据。系统按16位zjie编址。

输入:

 开 关:八个开关表示8位按16位zjie编址的存储器地址。

 按 钮:{[}可选{]}当按字读出时,最右按/不按选择显示大头/小头不同设计。

输出:

 发光管:自定。

 四数码:显示存储器读写地址(按zjie寻址)。

 八数码:显示32位存储器读写数据。

\hypertarget{ux5b9eux9a8cux8fc7ux7a0b}{%
\subsection{实验过程}\label{ux5b9eux9a8cux8fc7ux7a0b}}

首先编写 COE 文件如下:

\begin{verbatim}
memory_initialization_radix = 16;
memory_initialization_vector =
    00010203,
    04050607,
    08090A0B,
    0C0D0E0F,
    00102030,
    40506070,
    8090A0B0,
    C0D0E0F0
    ;
\end{verbatim}

接着利用 IP cores 生成一个 word size 为 32 bits,并且 words 的数量为 8
的 Block Memory

\begin{figure}
\centering
\includegraphics{img/2020-04-21-14-18-42.png}
\caption{Memory Settings}
\end{figure}

然后导入 COE

\begin{figure}
\centering
\includegraphics{img/2020-04-21-14-19-40.png}
\caption{COE Config}
\end{figure}

确认无误以后,编写控制代码如下:

\begin{Shaded}
\begin{Highlighting}[]
\NormalTok{Ram32b myram(.clka(clk), .wea(write_en), .dina(m_write_data), .addra(addr[}\DecValTok{3}\NormalTok{:}\DecValTok{1}\NormalTok{]), .douta(m_read_data));}

\KeywordTok{always}\NormalTok{ @(*) }\KeywordTok{begin}
    \KeywordTok{if}\NormalTok{ (is_little_endian) }\KeywordTok{begin}
        \KeywordTok{if}\NormalTok{ (addr[}\DecValTok{0}\NormalTok{] == }\DecValTok{0}\NormalTok{) }\KeywordTok{begin}
\NormalTok{            read_data_}\DecValTok{1}\NormalTok{ = m_read_data[}\DecValTok{15}\NormalTok{:}\DecValTok{0}\NormalTok{];}
        \KeywordTok{end} \KeywordTok{else} \KeywordTok{begin}
\NormalTok{            read_data_}\DecValTok{1}\NormalTok{ = m_read_data[}\DecValTok{31}\NormalTok{:}\DecValTok{16}\NormalTok{];}
        \KeywordTok{end}
    \KeywordTok{end} \KeywordTok{else} \KeywordTok{begin}
        \KeywordTok{if}\NormalTok{ (addr[}\DecValTok{0}\NormalTok{] == }\DecValTok{0}\NormalTok{) }\KeywordTok{begin}
\NormalTok{            read_data_}\DecValTok{1}\NormalTok{ = \{m_read_data[}\DecValTok{7}\NormalTok{:}\DecValTok{0}\NormalTok{], m_read_data[}\DecValTok{15}\NormalTok{:}\DecValTok{8}\NormalTok{]\};}
        \KeywordTok{end} \KeywordTok{else} \KeywordTok{begin}
\NormalTok{            read_data_}\DecValTok{1}\NormalTok{ = \{m_read_data[}\DecValTok{23}\NormalTok{:}\DecValTok{16}\NormalTok{], m_read_data[}\DecValTok{31}\NormalTok{:}\DecValTok{24}\NormalTok{]\};}
        \KeywordTok{end}
    \KeywordTok{end}
    \KeywordTok{if}\NormalTok{ (addr[}\DecValTok{0}\NormalTok{] == }\DecValTok{0}\NormalTok{) }\KeywordTok{begin}
\NormalTok{        m_write_data = \{m_read_data[}\DecValTok{31}\NormalTok{:}\DecValTok{16}\NormalTok{], write_data\};}
    \KeywordTok{end} \KeywordTok{else} \KeywordTok{begin}
\NormalTok{        m_write_data = \{write_data, m_read_data[}\DecValTok{15}\NormalTok{:}\DecValTok{0}\NormalTok{]\};}
    \KeywordTok{end}
\KeywordTok{end}
\end{Highlighting}
\end{Shaded}

测试代码,波形图呈现在实例分析小节。

编写 Top 模块,实现相应功能:

输入:

 开 关:八个开关表示8位按16位zjie编址的存储器地址。

 按 钮:{[}可选{]}当按字读出时,最右按/不按选择显示大头/小头不同设计。

输出:

 发光管:自定。

 四数码:显示存储器读写地址(按zjie寻址)。

 八数码:显示32位存储器读写数据。

形成最终项目:

\begin{figure}
\centering
\includegraphics{img/2020-04-21-14-31-27.png}
\caption{Project}
\end{figure}

\hypertarget{ux5b9eux4f8bux5206ux6790}{%
\subsection{实例分析}\label{ux5b9eux4f8bux5206ux6790}}

波形图如下:

\begin{figure}
\centering
\includegraphics{img/2020-04-21-14-04-23.png}
\caption{波形图}
\end{figure}

可以看出,在 big-endian 的模式下,原本的储存的数值 0x0203
输出方式成功改变成了 0x0302。接着可以观察到,当我改变 0x1 地址下的值为
0x4141 以后,当前地址的值的确被改变了。将读取地址变为 0x0
之后,可以看到储存在 0x0 的值并没有随之发生变化,因此可以证明对
half-word 的操作不会影响到储存器的其他值。

\end{document}
